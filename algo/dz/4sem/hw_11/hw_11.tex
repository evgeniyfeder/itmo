\documentclass[12pt, a4paper]{scrartcl}
\usepackage [utf8] {inputenc} 
\usepackage [english,russian] {babel}
\usepackage{indentfirst}
\usepackage{misccorr}
\usepackage{graphicx}
\usepackage{amsmath}
\usepackage [warn] { mathtext }

\begin{document}
	\LARGE{\textbf{Федер Евгений, Домашнее задание №11}}\par
	
	%%%%%%%%%%%%%%%%%%%%%%%%%%
	\emph{\textbf{Задание 1.}}\par
		Для удобности написания назовем функцию $f$
		$$f(s) = \sum\limits_{n = 1}^{\infty} n ^ {-s} $$
		Сделаем такое действие. Пройдемся по всем простым числам и на каждом шагу будем вычеркивать числа делящиеся на какое-то простое число.	Как это сделать? Пусть надо вычеркнуть число $p$.
		$$\frac{1}{p^s} * f(s) = \sum\limits_{n = 1}^{\infty} (p * n) ^ {-s} $$
		Это сумма чисел делящихся на p. Просто вычтем из исходной и получим сумму чисел не делящихся на $p$. \\
		Таким образом в предельном переходе получаем
		$$ \prod\limits_{p \in P}(1 - \frac{1}{p^s}) * f(s) = 1 $$
		$$ f(s) = \prod\limits_{p \in P}(1 - \frac{1}{p^s}) ^ {-1}$$
	
	%%%%%%%%%%%%%%%%%%%%%%%%%%
	\emph{\textbf{Задание 2.}}\par
		$$n = \prod p_i^{a_i}$$
		Каждая комбинация этих делителей дает какой-то новый делитель $n$. Поэтому можно расписать так(зная свойства мультипликативности для $\phi$): \\
		$\sum\limits_{d | p} \phi(d) = (\phi(1) + \phi(p_1) + \text{...} + \phi(p_1^{a_1})) * \text{...} * (\phi(1) + \phi(p_k) + ... + \phi(p_k^{a_k})) $ \\
		На лекции было, что $\phi(p^k) = p^k - p^{k - 1}$. Зная это каждая сумма полчается телескопической и получаем \\
		$\sum\limits_{d | p} \phi(d) = (\phi(1) + \phi(p_1) + \text{...} + \phi(p_1^{a_1})) * \text{...} * (\phi(1) + \phi(p_k) + ... + \phi(p_k^{a_k})) = \prod p_i^{a_i} = n$
		
	%%%%%%%%%%%%%%%%%%%%%%%%%%
	\emph{\textbf{Задание 3.}}\par
	Зафиксируем $n = c_1 \text{ и } p - q = c_2$, где $p > q$
	$$p = q + c_2$$
	$$n = c_1 = q * (q + c_2) \Rightarrow q^2 + c_2 * q - c_1 = 0$$
	$$ q = \frac{-c_2 + \sqrt{c_2^2 + 4 * c_1}}{2} $$
	
	Таким образом нам надо найти $p$ и $q$
	\begin{enumerate}
		\item $c_1$ дано
		\item $c_2$ перебираем за $O(|p - q|)$
		\item чекаем, что получили целый числа
		\item полилог берется из-за того, что константы могут быть большими и все арифметические действия будут делаться за полином от длины
	\end{enumerate}
\end{document}